%-------------------------------------------------------------------------------

\section{Convoluzione e sistemi LTI}
La convoluzione tra due segnali è equivalente al loro prodotto nel dominio 
della frequenza. Possiamo riscrivere la funzione di convoluzione dell'esercizio
precedente sfruttando questa relazione matematica:

\begin{equation}
	z(n) = x(n) \circledast y(n) \rightarrow 
	z(n) = F^{-1}(F[x(n)](f) \cdot F[y(n)](f))	
\end{equation}

%-------------------------------------------------------------------------------

\section{Implementazione DFT}
In MATLAB è possibile implementare le DFT e le IDFT come segue:

		
\begin{equation}
	x_{out}[k] = \begin{cases} 
	
	\frac{1}{N} \sum^{N-1}_{n=0} x_{in}[n] 
	e^{j 2 \pi k n / N } & k=0,...,N-1 \\ 
	
	\sum^{N-1}_{n=0} x_{in}[n] 
	e^{- j 2 \pi k n / N } & k=0,...,N-1 
	
	\end{cases}
\end{equation}

dove N è la lunghezza del segnale $x_{in}[n]$.

%-------------------------------------------------------------------------------